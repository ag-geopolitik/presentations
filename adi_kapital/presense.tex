--

Karl Marx ist studierter Philosoph. Vielleicht wurde aus ihm, aus 
heutiger Sicht eher ein politischer Philosoph, denn Philosophie
hat in der Zeit nach ihm ihre Unschuld verloren bzw. aufgegeben.

Werkzeug, Aufgabe und Ausdrucksform in einem.

--

Soziologie
==========

De Omnibus Dubitandum (We ought to question everything)
 — Karl Marx’s favourite motto.

Heute zählt sein Werk zu den Grundlagentexten der
Soziologie. Er ist Referenz für eine Betrachtungsweise
von Gesellschaft, die Gruppen mit einander entgegengesetzten
Interessen und die daraus hervorgehenden Kämpfe und Konflikte
untersucht.

Sein Lebenswerk, die sozio-ökonomische Analyse der kapitalistischen
Gesellschaft, die Darstellung der Faktoren, welche zu ihrer
Entstehung führten und die Prognosen zur deren weiteren Entwicklung,
werden heute als wichtiger Beitrag zur Entwicklung der Soziologie
anerkannt.

--

Die Politikwissenschaft knüpft ebenfalls zum Beispiel bei
Theorien der internationalen Beziehungen an die Kritik der
Politischen Ökonomie an. 

--

Doch wie sehen wir ihn heute als Ökonom?

  * Die zentralisierte Planwirtschaft die in sozialistischen
    Staaten im 20. Jahrhundert praktiziert wurde, hatte neben
    einzelnen Erfolgen, langfristig wenige Vorteile im Vergleich
    zur kapitalistischen Marktwirtschaft.

  * Seine Arbeitswertheorie wird als ideologisch begründet angesehen
    und sie kann nicht die Entstehung von Marktpreisen erklären.

-- 

Marx war Humanist, aber kein Utopist, denn Utopia liegt im Nirgendwo.

-- 

Für Marx war die Überwindung des Kapitalismus eine Selbstverständlichkeit.
Wenn die Menschen die Chance bekommen, diesen Verhältnissen zu entkommen
so gab es für ihn keinen Zweifel, dann würden sie diese ergreifen.

Die realen Arbeits- und Lebensbedingungen des Proletariats ließen ihn in
dieser Frage zumindest ziemlich sicher sein.

--

Unter Diktatur des Proletariats ist die Selbstbesimmung der Arbeiter
über ihre Arbeit zu verstehen. Ein Zustand der zu Lebzeiten von Karl
Marx während der Errichtung der Pariser Kommune angestrebt wurde. 

Wann und wie hat der Begriff der Diktatur seine heutige Bedeutung erhalten?

--

Quellen:

Marx and Marxism - Peter Worsley
Der Staat im globalen Kapitalismus - Samuel Decker(2013)
